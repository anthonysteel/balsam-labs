\documentclass[12pt, letterpaper, twoside]{article}
\usepackage[utf8]{inputenc}

\title{Balsam Labs 2021\\ End of Year}
\author{Anthony Steel}
\date{December 2021}

\begin{document}

\begin{titlepage}
\maketitle
\end{titlepage}}
\section{Projects}
This year the lab pursued two projects:
\begin{enumerate}
\item A blender that sounds good
\item A robot that cooks for people
\end{enumerate}
The blender was started to provide a source of income for the
lab by 2024.
\section{Blender}
Why are blenders loud? A common answer would be
that they require a reasonably powerful motor.
By far the most common power drive for a blender
is a univerisal AC brushed motor. This is a motor which uses
graphite sticks called brushes to send the correct electrical signals to
the motor to induce a rotating magnetic field. The process of
sending these signals is known as commutation and because brushed
motors use physical contact to accomplish it, they are inherently
sound producing. One can remove this source of sound by using a
permanent magnet (PM) motor. However, the cost of the blender is increased
for two reasons. Firstly, the commutation must now be accomplished
through electrical feedback and a motor controller. Secondly,
because the speed of the motor must vary, the commutation signals
must be generated by an inverter from a DC power supply.

In January 2021, a simple prototype was developed to test the sound
reduction produced by using a PM motor instead of a universal brushed
one. The prototype is pictured in Figure Blah. The prototype contained:
\begin{enumerate}
\item T-Motor
\item T-Motor
\item An external DC power supply.
\end{enumerate}
Though this resulted in approximately 20dBA reduction on a sound meter at three feet,
the sound being produced was still loud and unpleasent\footnote{It would be
expensive and time consuming to build an anachoic chamber with exhaustive
sound metering to produce quantitative sound measurment. Sound reduction
measurement has been performed with a cheap handheld sound meter and subjective
experience by ear. The final intent for the blender is to be a consumer product
so the most important measurement is subjective perception of the sound.}.

The primary source of persistent sound was a result of a process
called magnetostriction within the motor. Broadly defined, magnetostriction
is a property of magentic and ferromagnetic materials that causes them to
change their dimensions when they are magnetized. This change in dimensions
causes resonance vibrations with the motor chassis and surrounding material
which leads to audible sound. In order to reduce the sound produced by
magnetostriction there are two options.
\begin{enumerate}
\item Engineer a new ferromagnetic material which inherently undergoes less
  magnetostriction from an applied current.
\item Produce the magnetic field which causes the least audible amount of
  magnetostriction.
\end{enumerate}
Because the first option was deemed out of scope, the second was pursued.

Through experimentation, it was discovered that the audible noise generated
by the motor was reduced when a smooth sine wave was sent as a commutation
signal. The means of producing this signal was by a method known as Field
Oriented Control (FOC).
\end{document}
