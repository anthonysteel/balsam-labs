\documentclass[12pt, letterpaper, twoside]{article}
\usepackage[utf8]{inputenc}
\title{Balsam Labs 2021\\ End of Year}
\author{Anthony Steel}
\date{December 2021}
\begin{document}
\begin{titlepage}
\maketitle
\end{titlepage}}
\section{Projects}
This year the lab pursued two projects:
\begin{enumerate}
\item A blender that sounds good
\item A robot that cooks for people
\end{enumerate}
The blender was started to provide a source of income for the
lab by 2024.
\section{The Blender}
How do you make a blender which produces a quiet pleasant sound?
The blender will require a motor and there are many different types
of motors to choose from. Among the existing toplogies, we would
like to select a motor which makes no noise, however this is not
possible. The reason is Maxwell's stress-energy tensor. All magnetic
and ferromagnetic materials exhibit a property called magnetostriction,
by which their dimensions change under the presence of an applied magnetic
field. Atomically, this is caused by magnetic domains in the material
behaving as dipoles. If you imagine a dipole idealized as an elliptical
bar magnetic, when the magnetic rotates under the presence of an applied
field, it acts like a cam and the dimension of the material changes at
the boundary. This process is visualized crudely in Figure #.

Most motor are constructed from a sheets of silicon steel (steel that has
been mixed with trace amounts of silicon) in order to reduce the effects
of magnetostriction and other losses such as eddy currenty. The sheets of
steel are die cut in the shape of the stator and laminated together with
a resin to further reduce eddy current losses. There are two ways to
reduce magnetostriction:
\begin{enumerate}
\item Discover an improved ferromagnetic alloy which exhibits less magnetostriction
\item Reduce the sound produces by an magentrostriction in an existing material by
  controlling the magnetic field.
\end{enumerate}
The lab has chosen to persue the second option. By experimenting with different
types of motor controllers the current waveform which is applied to the motor
has a dramatic effect on the sound the motor produces.
\end{document}
