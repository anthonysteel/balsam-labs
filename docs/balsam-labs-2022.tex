\documentclass[12pt, letterpaper, twoside]{article}
\usepackage[utf8]{inputenc}
\usepackage{amsthm}
\usepackage[english]{babel}
\theoremstyle{definition}
\newtheorem{definition}{Definition}[section]
\title{Balsam Labs\\ Research Program 2022}
\author{Anthony Steel}
\date{January 2022}
\begin{document}
\begin{titlepage}
\maketitle
\end{titlepage}
\section{Teaching a robot to cook}
Imagine that we want to build a robot that can be installed
in a household kitchen to cook meals from raw ingredients and
we are not constrained by cost. In order to prepare the meal
we require a means of manipulating ingredients and tools. Humans
use their hands for this. You could argue that a special
purpose manipulator should be built for the task but consider the following
argument: The space of possible meals that a robot can prepare
is defined by the manipulator it uses. If we want to span the space
of all meals humans cook, the manipulator design approaches the human hand.
The number of digits on your manipulator may not need to be exactly five, but
for simplicity, lets assume we will use a biomimetic robotic hand with
five digits. We can allow the manipulator to achieve an arbitrary pose
in the kitchen by means of a robotic arm with the necessary degrees of freedom
and a gantry mechanism if required.

At this point, we have introduced nothing that is not already commercially
available. There are a number of appropriate robotic arms (Universal Robots'
UR5 or Kuka's IIWA) and biomimetic robotic hands (Shadow's Shadow Hand).
So why does such a robot not exist? The problem lies in providing the
appropriate commands to the manipulator. I will call the process of sending commands
to a humanoid manipulator to perform an arbitrary task accomplishable by
humans dexterous manipulation.
\begin{definition}[Dexterous Manipulation]
  Sending commands to a humanoid manipulator to perform an arbitrary task
  accomplishable by humans.
\end{definition}
Dexterous manipulation, at the time of writing, is an unsolved
problem in robotics. For example, it is not known how to produce a set of
commands that will allow a robotic hand to peel an onion, julienne carrots, or
roll sushi. To determine how we can send these commands, let us consider
how humans perform such a task.

Humans are able to perform the above tasks because of our sense of touch.
You may argue that vision is of primary importance but consider this:
Many skilled tasks can be performed without vision, such as cutting onions,
peeling carrots, tying knots etc. In fact, during such tasks most of the object
is occluded by the hands, even when vision is used. Human depth
perception is actually remarkably bad without feedback from touch, as can
be demonstrated by experiments in VR. An argument often posed to me
against touch is the following: it is possible to teleoperate a robot to
perform complicated tasks such as surgery and the robot does not require
a sense of touch. However, the process in the surgeons brain
by which the commands to the joystick are being produced are using
touch sensation of the joystick as an input. I would imagine the surgeon
would be hesitant to perform such a surgery with aneasthetized hands.

If we believe that touch, rather than traditional vision, is the appropriate
sensory modality for manipulation, how do we use it to produce the necessary
joint commands for dexterous manipulation?

\section{Human Touch}
There are a number of neuron types collectively called Low Threshold Mechanoreceptors
or LTMRs. They are subdivided in to 7 categories:
\begin{enumerate}
\item Slow Activation I (SAI-LTMR): Indentation
\item Slow Activation II (SAII-LTMR): Stretch
\item Rapid Activation I (RAI-LTMR): Skin movement, hair follicle deflection
\item Rapid Activation II (RAII-LTMR): Vibration
\item A-delta (Ad-LTMR): Hair follicle deflection
\item C (C-LTMR): Hair follicle deflection
\item High Threshold Mechanoreceptors (HTMRs): Noxious mechanical
\end{enumerate}

\section{Algorithm}

\end{document}
